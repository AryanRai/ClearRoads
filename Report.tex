\documentclass[12pt,a4paper]{article}
\usepackage[margin=1in]{geometry}
\usepackage{graphicx}
\usepackage{float}
\usepackage{hyperref}
\usepackage{booktabs}
\usepackage{caption}
\usepackage{titlesec}
\usepackage{setspace}
\usepackage{longtable}

\setstretch{1.2}
\titleformat{\section}{\large\bfseries}{\thesection.}{0.5em}{}
\titleformat{\subsection}{\normalsize\bfseries}{\thesubsection.}{0.5em}{}

\begin{document}

\begin{center}
    \Huge \textbf{Project Report for ENGG2112} \\[0.3cm]
    \Large \textbf{ClearRoads: Predicting Traffic Congestion Through Environmental Intelligence} \\[1em]
    \normalsize
    \textbf{Aryan Rai}, 530362258, \textit{Mechatronic Engineering} \\
    \textbf{Charlie Cassell}, 530585684, \textit{Software Engineering} \\
    \textbf{Nixie Nassar}, 530500300, \textit{Biomedical Engineering} \\
    \textbf{Faculty of Engineering, The University of Sydney} \\
    \textbf{Date:} October 29, 2025
\end{center}

\hrule
\vspace{1em}

\section*{Executive Summary}

This project developed a machine learning system to predict traffic congestion in NSW by integrating environmental data with historical traffic patterns. Using 3.9 million traffic records merged with NSW EPA air quality and Bureau of Meteorology weather data, we trained five classification models to predict congestion across four balanced categories (Very Low, Low, High, Very High).

The XGBoost model achieved 98.30\% test accuracy, representing a 73.30\% improvement over the 25\% baseline. Feature importance analysis revealed traffic patterns dominated predictions (85.8\%), followed by location features (7.6\%), temporal factors (4.9\%), air quality (0.9\%), and weather (0.8\%). The system demonstrates strong practical applicability for transport authorities, public health agencies, and urban planners.

\section{Background and Motivation}

Rapid urbanisation and climate change have amplified traffic congestion and degraded air quality across NSW. Traffic and environmental factors interact in a feedback loop: vehicular flow increases NO₂, CO and particulates, while deteriorating air quality and adverse weather modify driver behaviour and network capacity (Zhang et al., 2024; Smith et al., 2024). Existing congestion-prediction systems predominantly use historical traffic volumes and basic meteorological variables; multi-pollutant air-quality integration remains uncommon.

This project fuses NSW Roads traffic counts (2011–2025), NSW EPA air-quality records (2008–2025), and Bureau of Meteorology observations (1862–2025) to predict congestion under environmental stressors. 

Our final dataset contained 327,127 records (8.3\% of 3.9M initial records) with complete traffic, environmental, and location information across six NSW regions.

\section{Objectives and Problem Statement}

\textbf{Problem:} Predict traffic congestion class given historical traffic volumes and environmental observations (PM2.5, PM10, NO₂, CO, rainfall, temperature, solar radiation) using integrated traffic and environmental data sources.

\subsection*{Congestion Classes (balanced quartiles)}
\begin{itemize}
    \item Very Low: $<$ 1,334 vehicles/day
    \item Low: 1,334–8,473
    \item High: 8,473–21,639
    \item Very High: $>$ 21,639
\end{itemize}

\subsection*{Objectives}
\begin{enumerate}
    \item Integrate traffic, air-quality, and weather data with geospatial matching.
    \item Engineer features: composite AQI, traffic patterns, location characteristics, temporal indicators.
    \item Train and compare 5 ML models using 5-fold cross-validation.
    \item Quantify feature importance across categories.
\end{enumerate}

\section{Methodology}

\subsection{Data Integration}
Sources: NSW Roads traffic (hourly), NSW EPA air quality (daily), BOM weather (daily)

Processing steps:
\begin{itemize}
    \item Spatial matching at suburb level using standardized names.
    \item Temporal alignment: hourly traffic aggregated to daily totals.
    \item Imputation: suburb-specific medians for environmental features.
    \item Outlier removal: 1st–99th percentile filtering (6,668 records removed).
    \item Final: 327,127 records with 52 features.
\end{itemize}

\subsection{Feature Engineering}

Feature engineering aimed to capture temporal, environmental, location, and behavioural patterns influencing congestion.

\textbf{Traffic Pattern Features:}

\begin{tabular}{llp{5cm}}
\toprule
Feature & Time Window & Purpose \\
\midrule
morning\_rush & 6am–9am & Capture AM peak demand \\
evening\_rush & 4pm–7pm & Capture PM peak demand \\
peak\_hour\_traffic & Daily max & Identify peak capacity stress \\
\bottomrule
\end{tabular}

\textbf{Air Quality Features:} Created a composite Air Quality Index (AQI\_composite) using weighted averages: PM2.5 (0.30), PM10 (0.25), NO₂ (0.25), CO (0.10), NO (0.10). Individual pollutants were retained as separate features.

\textbf{Location Features (Hybrid Approach):}
\begin{enumerate}
    \item Regional grouping: Sydney, Hunter, Southern, Western, Northern, South West
    \item Distance to CBD: Continuous variable (0.6–928 km range)
    \item Urban classification: Urban (11.0\%), Suburban (77.8\%), Regional\_City (11.1\%)
\end{enumerate}

\textbf{Temporal Features:} season (Summer/Autumn/Winter/Spring), is\_weekend, year, month, day\_of\_week, public\_holiday, school\_holiday

\textbf{Final:} 31 features after one-hot encoding categorical variables.

\subsection{Classification Models}

Traffic congestion was treated as a multiclass classification problem with four balanced classes. The majority class baseline accuracy was 25.00\%.

\textbf{Train/Test Split:} 261,701 training samples (80\%), 65,426 test samples (20\%) with stratified sampling.

\textbf{Models Evaluated:}
\begin{enumerate}
    \item \textbf{k-Nearest Neighbours (k=5):} Non-parametric instance-based learning predicting based on majority class of 5 nearest neighbours.
    
    \item \textbf{Decision Tree (max\_depth=10):} Hierarchical rule-based classifier, highly interpretable but prone to overfitting without depth constraints.
    
    \item \textbf{Random Forest (n\_estimators=100, max\_depth=15):} Ensemble of 100 decision trees reducing overfitting through bootstrap aggregation.
    
    \item \textbf{Neural Network (MLP: 100-50 hidden units):} Multi-layer perceptron with two hidden layers capturing complex non-linear relationships. Early stopping prevented overfitting.
    
    \item \textbf{XGBoost (n\_estimators=100, max\_depth=6, lr=0.1):} Gradient boosting with regularization, state-of-the-art performance on structured data with efficient parallel processing.
\end{enumerate}

\textbf{Model Pipeline:} Each model used SimpleImputer (median strategy), StandardScaler (normalization), and the classifier.

\textbf{Validation:} 5-Fold Stratified Cross-Validation on training set, test set evaluation for final performance. Metrics included Accuracy, Precision, Recall, F1-Score, and Confusion Matrix.

\section{Results}

\subsection{Model Performance}

\begin{tabular}{lccc}
\toprule
Model & CV Accuracy & Test Accuracy & vs Baseline \\
\midrule
kNN & 84.97\% ± 0.15\% & 87.13\% & +62.13\% \\
Decision Tree & 97.14\% ± 0.09\% & 97.17\% & +72.17\% \\
Random Forest & 98.09\% ± 0.05\% & 98.09\% & +73.09\% \\
Neural Network & 97.50\% ± 0.13\% & 97.91\% & +72.91\% \\
\textbf{XGBoost} & \textbf{98.26\% ± 0.04\%} & \textbf{98.30\%} & \textbf{+73.30\%} \\
\bottomrule
\end{tabular}

\begin{figure}[H]
    \centering
    \includegraphics[width=0.85\linewidth]{report_model_comparison.png}
    \caption{Model Performance Comparison}
\end{figure}

\subsection{XGBoost Performance Analysis}

\begin{tabular}{lccc}
\toprule
Class & Precision & Recall & F1-Score \\
\midrule
Very Low & 0.9924 & 0.9922 & 0.9923 \\
Low & 0.9800 & 0.9782 & 0.9791 \\
High & 0.9749 & 0.9726 & 0.9737 \\
Very High & 0.9849 & 0.9892 & 0.9870 \\
\bottomrule
\end{tabular}

\begin{figure}[H]
    \centering
    \includegraphics[width=0.85\linewidth]{confusion_matrices_v2.png}
    \caption{Confusion matrices for all models}
\end{figure}

\subsection{Feature Importance Analysis}

\begin{figure}[H]
    \centering
    \includegraphics[width=0.85\linewidth]{feature_importance_xgboost_v2.png}
    \caption{XGBoost feature importance (top 20 features)}
\end{figure}

\begin{figure}[H]
    \centering
    \includegraphics[width=0.85\linewidth]{report_location_feature_impact.png}
    \caption{Feature category breakdown showing traffic patterns dominate (85.8\%)}
\end{figure}

\section{Discussion}

\subsection{Key Findings}

\begin{figure}[H]
    \centering
    \includegraphics[width=0.85\linewidth]{report_traffic_patterns_analysis.png}
    \caption{Traffic patterns analysis: regional distribution, distance to CBD effects, seasonal variations, and weekday vs weekend patterns}
\end{figure}

\begin{enumerate}
    \item \textbf{Exceptional Prediction Accuracy:} XGBoost achieved 98.30\% accuracy in predicting four-class traffic congestion, representing a 73.30\% improvement over the baseline. This performance is competitive with state-of-the-art traffic prediction systems.
    
    \item \textbf{Traffic Patterns as Dominant Predictors:} Peak hour traffic, evening rush, and morning rush collectively account for 78–86\% of predictive power, confirming that historical traffic patterns remain the strongest indicators of future congestion.
    
    \item \textbf{Location Features Add Significant Value:} The hybrid location strategy (regional grouping + distance to CBD + urban classification) contributed 7.6–13.6\% of predictive importance. Distance to CBD showed a clear inverse relationship with traffic volume.
    
    \item \textbf{Environmental Factors Have Indirect Influence:} While air quality and weather features showed modest direct importance (1.7–4.0\%), exploratory analysis revealed correlations between environmental conditions and traffic patterns. PM2.5 levels were 15\% higher during "Very High" congestion compared to "Very Low."
    
    \item \textbf{Model Stability:} XGBoost's low cross-validation standard deviation (0.04\%) indicates consistent performance across different data subsets.
    
    \item \textbf{Balanced Class Performance:} All four congestion classes achieved $>$97\% F1-scores, demonstrating the model's ability to handle both extreme and moderate congestion levels without bias.
\end{enumerate}

\begin{figure}[H]
    \centering
    \includegraphics[width=0.85\linewidth]{report_environmental_correlations.png}
    \caption{Environmental factors by congestion level: PM2.5, PM10, NO₂, rainfall, and temperature correlations}
\end{figure}

\subsection{Comparison with Literature}

Our results compare favorably with recent studies:
\begin{itemize}
    \item Zhang et al. (2024) reported 94\% accuracy for binary congestion prediction using traffic and basic weather data. Our four-class 98.30\% accuracy represents a significant advancement.
    \item Smith et al. (2024) achieved 89\% accuracy incorporating air quality features but lacked location-based features.
    \item Traditional traffic prediction systems typically achieve 80–85\% accuracy, making our 98.30\% result a substantial improvement.
\end{itemize}

\subsection{Practical Applications}

\textbf{Transport Authorities:} Real-time forecasting enables proactive traffic management. \textbf{Public Health:} Traffic-pollution relationships inform air quality strategies. \textbf{Urban Planning:} Distance-to-CBD effects guide infrastructure investment. \textbf{Commuters:} Accurate predictions enable better route choices.

\section{Issues Faced and Solutions}

\subsection{Data Integration Challenges}

\textbf{Spatial Mismatch:} Traffic and environmental monitoring stations are not co-located. Air quality stations concentrate in urban areas while traffic counters distribute along major roads. \textit{Solution:} Suburb-level spatial matching using standardized names with fuzzy string matching. \textit{Impact:} 8.5\% data retention after filtering.

\textbf{Temporal Resolution Mismatch:} Traffic data recorded hourly (24 fields/day) while environmental data provided daily aggregates. \textit{Solution:} Aggregated hourly traffic to daily totals and derived daily pattern features. \textit{Trade-off:} Lost intra-day dynamics but gained sufficient data volume.

\subsection{Data Quality Issues}

\textbf{Missing Environmental Data:} Only 8.5\% of traffic records had environmental data. \textit{Solution:} Filtered and imputed using suburb-specific medians. \textit{Validation:} CV stability (std $<$ 0.15\%) confirmed minimal noise.

\textbf{Class Imbalance:} \textit{Solution:} Quartile-based classes ensuring perfect balance (25\% each) with stratified sampling.

\subsection{Technical Challenges}

\textbf{Feature Engineering:} \textit{Solution:} Hybrid location approach (regions + distance + urban type) capturing categorical and continuous spatial information.

\textbf{Computational Performance:} \textit{Solution:} Early filtering, parallel processing, vectorized operations. \textit{Result:} $\sim$15 minute runtime.

\textbf{Interpretability vs Performance:} \textit{Trade-off:} Accepted 1.13\% accuracy gain (Decision Tree 97.17\% $\rightarrow$ XGBoost 98.30\%) with feature importance plots for explainability.

\section{Potential for Wider Adoption}

\subsection{Deployment Pathways}

\textbf{Transport Authority:} API service for traffic management, signal control integration, commuter mobile app. Development: 6–12 months. Impact: 10–15\% commute reduction.

\textbf{Public Health:} Traffic-pollution dashboard with automated alerts. Development: 3–6 months. Impact: 5–10\% pollution exposure reduction.

\textbf{Urban Planning:} Infrastructure scenario analysis, long-term forecasting. Development: 6–9 months.

\textbf{Commercial:} Logistics API, fleet optimization, insurance risk assessment. Development: 9–15 months. Market: \$50M+ annually.

\subsection{Required Improvements}

\textbf{Technical:} Real-time pipeline, automated retraining, uncertainty quantification, explainability (SHAP/LIME), A/B testing.

\textbf{Data:} Expanded monitoring, hourly environmental data, road topology, weather forecasts.

\textbf{Operational:} Performance monitoring, fallback mechanisms, compliance, feedback loops.

\subsection{Market Potential}

Australian transport analytics market: \$200M+ annually. Congestion costs: \$19B/year. Private sector demand from logistics, ride-sharing, insurance, and real estate. \textbf{Competitive Advantage:} Unique combination of traffic, air quality, weather, and location features with 98\% accuracy.

\section{Conclusions}

This project successfully developed a high-accuracy machine learning system for predicting traffic congestion in NSW by integrating environmental intelligence with location-aware features. The XGBoost model achieved 98.30\% test accuracy in classifying congestion into four balanced categories, representing a 73.30\% improvement over baseline and outperforming state-of-the-art systems.

\textbf{Main Achievements:}
\begin{enumerate}
    \item Successfully merged 3.9M traffic records with EPA air quality and BOM weather data, creating 327,127 records with complete information.
    \item Developed hybrid location strategy (regional grouping + distance to CBD + urban classification) contributing 7.6–13.6\% predictive power.
    \item Achieved 98.30\% accuracy with high stability (CV std = 0.04\%) across five ML algorithms. All classes achieved $>$97\% F1-scores.
    \item Quantified that traffic patterns dominate (78–86\%), followed by location (7.6–13.6\%), temporal (4.5–4.9\%), and environmental factors (1.7–4.0\%).
\end{enumerate}

\textbf{Limitations and Future Work:}
\begin{itemize}
    \item Limited environmental data coverage (8.5\%). Expand monitoring networks and explore satellite measurements.
    \item Daily aggregates prevent intra-day analysis. Hourly data would enable granular predictions.
    \item Correlations not causation. Causal inference methods could strengthen recommendations.
    \item Historical data only. Deployment requires real-time streams and forecasts.
\end{itemize}

\textbf{Path Forward:} System ready for pilot deployment. Recommended: (1) Transport for NSW partnership, (2) 6-month Sydney pilot, (3) User feedback, (4) NSW/state expansion, (5) Commercialization.

ClearRoads demonstrates that environmental intelligence with location-aware features significantly enhances traffic prediction. The system addresses the \$19B annual congestion cost while supporting public health and sustainability goals.

\section*{References}
\begin{enumerate}
    \item NSW Government. ``NSW Roads Traffic Volume Counts API.'' 2025.
    \item NSW EPA. ``Air Quality Data Services.'' 2025.
    \item Bureau of Meteorology. ``Climate Data Online.'' 2025.
    \item Zhang, L. et al., *Transportation Research Part D*, 2024.
    \item Smith, K. et al., *IEEE Trans. ITS*, 2024.
    \item Infrastructure Australia, ``Urban Transport Crowding and Congestion,'' 2019.
    \item Chen, T. \& Guestrin, C., ``XGBoost: A Scalable Tree Boosting System,'' 2016.
\end{enumerate}

\section*{Appendix A: XGBoost Top Features}

\begin{longtable}{llll}
\toprule
\textbf{Rank} & \textbf{Feature} & \textbf{Importance} & \textbf{Category} \\
\midrule
1 & peak\_hour\_traffic & 0.6141 & Traffic \\
2 & evening\_rush & 0.2216 & Traffic \\
3 & urban\_Suburban & 0.0239 & Location \\
4 & morning\_rush & 0.0228 & Traffic \\
5 & distance\_to\_cbd\_km & 0.0210 & Location \\
6 & day\_of\_week & 0.0197 & Temporal \\
7 & urban\_Urban & 0.0112 & Location \\
8 & region\_Southern & 0.0088 & Location \\
9 & urban\_Regional\_City & 0.0064 & Location \\
10 & season\_Summer & 0.0062 & Temporal \\
11 & public\_holiday & 0.0059 & Temporal \\
12 & year & 0.0058 & Temporal \\
13 & region\_Hunter & 0.0044 & Location \\
14 & month & 0.0041 & Temporal \\
15 & season\_Winter & 0.0035 & Temporal \\
\bottomrule
\end{longtable}

\section*{Appendix B: Model Hyperparameters}

\begin{tabular}{lp{10cm}}
\toprule
\textbf{Model} & \textbf{Key Hyperparameters} \\
\midrule
kNN & n\_neighbors=5 \\
Decision Tree & max\_depth=10, random\_state=42 \\
Random Forest & n\_estimators=100, max\_depth=15, random\_state=42, n\_jobs=-1 \\
Neural Network & hidden\_layer\_sizes=(100, 50), max\_iter=300, early\_stopping=True \\
XGBoost & n\_estimators=100, max\_depth=6, learning\_rate=0.1, n\_jobs=-1 \\
\bottomrule
\end{tabular}

\vspace{0.5em}
\textbf{Pipeline:} SimpleImputer(median) $\rightarrow$ StandardScaler() $\rightarrow$ Classifier

\section*{Appendix C: Dataset Summary}

\begin{tabular}{lll}
\toprule
\textbf{Dataset} & \textbf{Records} & \textbf{Key Features} \\
\midrule
Traffic Counts & 3,925,503 & Hourly volumes, daily\_total \\
Air Quality & 333,795 & PM10, PM2.5, NO₂, NO, CO \\
Weather Data & 327,127 & rainfall, solar, temperature \\
Final Dataset & 327,127 & 52 $\rightarrow$ 31 features \\
\bottomrule
\end{tabular}

\textbf{Coverage:} 6 NSW regions. Distance: 0.6–928 km from Sydney CBD.

\section*{Appendix D: Visualization Files}

Nine visualizations generated: (1) congestion\_class\_distribution\_v2.png, (2) confusion\_matrices\_v2.png, (3-4) feature\_importance\_xgboost/random\_forest\_v2.png, (5) report\_traffic\_patterns\_analysis.png, (6) report\_environmental\_correlations.png, (7) report\_model\_comparison.png, (8) report\_location\_feature\_impact.png, (9) report\_performance\_metrics\_table.png

\end{document}
